\documentclass[fleqn]{article}
\usepackage[nodisplayskipstretch]{setspace}
\usepackage{amsmath, nccmath, bm}
\usepackage{amssymb}
\usepackage{enumitem}

\newcommand{\zerodisplayskip}{
	\setlength{\abovedisplayskip}{0pt}%
	\setlength{\belowdisplayskip}{0pt}%
	\setlength{\abovedisplayshortskip}{0pt}%
	\setlength{\belowdisplayshortskip}{0pt}%
	\setlength{\mathindent}{0pt}}
	
\title{Homework 1}
\author{Owen Sowatzke}
\date{February 7, 2024}

\begin{document}

	\offinterlineskip
	\setlength{\lineskip}{12pt}
	\zerodisplayskip
	\maketitle
	
	\begin{enumerate}
		\item Let $\mathbf{\Sigma_1}$ and $\mathbf{\Sigma_2}$ be two symmetric, positive definite matrices. Let $\mathbf{\Delta}$ be a diagonal matrix containing the eigenvalues of $\mathbf{\Sigma_1^{-1}}\mathbf{\Sigma_2}$. Let $\mathbf{A}$ contain the corresponding eigenvector of $\mathbf{\Sigma_1^{-1}}\mathbf{\Sigma_2}$ as its rows in the same order as the eigenvalue order in $\mathbf{\Sigma}$. Prove that
		
		\begin{equation*}
			\mathbf{A\Sigma_1A^T} = \mathbf{I}\quad\text{(Identity)}
		\end{equation*}
		
		\begin{equation*}
			\mathbf{A\Sigma_2A^T} = \mathbf{\Delta}\quad\text{(Diagonal)}
		\end{equation*}
		
		\begin{equation*}
			\mathbf{A\Sigma_1^{-1}\Sigma_2A^T = \Delta}
		\end{equation*}
		
		\begin{equation*}
			\mathbf{\Sigma_2A^T = \Sigma_1A^T\Delta}
		\end{equation*}
		
		\item Consider two symmetric positive semi-definite matrices $\mathbf{\Sigma_1}$ and $\mathbf{\Sigma_2}$ of size $N \times N$. Let $\mathbf{P}$ be a $N \times N$ matrix such that
		
		\begin{equation*}
			\mathbf{P}(\mathbf{\Sigma_1} + \mathbf{\Sigma_2})\mathbf{P^T} = \mathbf{I}
		\end{equation*}
		
		If we define $\mathbf{A} = \mathbf{P}\mathbf{\Sigma_1}\mathbf{P^T}$ and $\mathbf{B} = \mathbf{P}\mathbf{\Sigma_2}\mathbf{P^T}$, then determine the properties of the eigenvalues of $\mathbf{A}$ and $\mathbf{B}$.
		
		\item skip
		
		\item Expand $(\mathbf{A} + \mathbf{B})(\mathbf{A} - \mathbf{B})$ and $(\mathbf{A} - \mathbf{B})(\mathbf{A} + \mathbf{B})$. Are these expressions the same? If not, why not?
		
		\begin{equation*}
		(\mathbf{A} + \mathbf{B})(\mathbf{A} - \mathbf{B}) = \mathbf{A^2} + \mathbf{B}\mathbf{A} - \mathbf{A}\mathbf{B} - \mathbf{B^2}
		\end{equation*}
		
		\begin{equation*}
		(\mathbf{A} - \mathbf{B})(\mathbf{A} + \mathbf{B}) = \mathbf{A^2} - \mathbf{B}\mathbf{A} + \mathbf{A}\mathbf{B} - \mathbf{B^2}
		\end{equation*}
		
		The above expressions are not the same because matrix multiplication is not always commutative ($\mathbf{AB} \neq \mathbf{BA}$).
	\end{enumerate}
\end{document}